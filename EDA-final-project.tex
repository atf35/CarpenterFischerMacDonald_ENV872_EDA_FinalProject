% Options for packages loaded elsewhere
\PassOptionsToPackage{unicode}{hyperref}
\PassOptionsToPackage{hyphens}{url}
%
\documentclass[
  12pt,
]{article}
\title{Seasonal Salt Content in the Neuse River}
\usepackage{etoolbox}
\makeatletter
\providecommand{\subtitle}[1]{% add subtitle to \maketitle
  \apptocmd{\@title}{\par {\large #1 \par}}{}{}
}
\makeatother
\subtitle{\url{https://github.com/atf35/CarpenterFischerMacDonald_ENV872_EDA_FinalProject}}
\author{Atalie Fischer, Kathlyn MacDonald, Jack Carpenter}
\date{}

\usepackage{amsmath,amssymb}
\usepackage{lmodern}
\usepackage{iftex}
\ifPDFTeX
  \usepackage[T1]{fontenc}
  \usepackage[utf8]{inputenc}
  \usepackage{textcomp} % provide euro and other symbols
\else % if luatex or xetex
  \usepackage{unicode-math}
  \defaultfontfeatures{Scale=MatchLowercase}
  \defaultfontfeatures[\rmfamily]{Ligatures=TeX,Scale=1}
  \setmainfont[]{Times New Roman}
\fi
% Use upquote if available, for straight quotes in verbatim environments
\IfFileExists{upquote.sty}{\usepackage{upquote}}{}
\IfFileExists{microtype.sty}{% use microtype if available
  \usepackage[]{microtype}
  \UseMicrotypeSet[protrusion]{basicmath} % disable protrusion for tt fonts
}{}
\makeatletter
\@ifundefined{KOMAClassName}{% if non-KOMA class
  \IfFileExists{parskip.sty}{%
    \usepackage{parskip}
  }{% else
    \setlength{\parindent}{0pt}
    \setlength{\parskip}{6pt plus 2pt minus 1pt}}
}{% if KOMA class
  \KOMAoptions{parskip=half}}
\makeatother
\usepackage{xcolor}
\IfFileExists{xurl.sty}{\usepackage{xurl}}{} % add URL line breaks if available
\IfFileExists{bookmark.sty}{\usepackage{bookmark}}{\usepackage{hyperref}}
\hypersetup{
  pdftitle={Seasonal Salt Content in the Neuse River},
  pdfauthor={Atalie Fischer, Kathlyn MacDonald, Jack Carpenter},
  hidelinks,
  pdfcreator={LaTeX via pandoc}}
\urlstyle{same} % disable monospaced font for URLs
\usepackage[margin=2.54cm]{geometry}
\usepackage{longtable,booktabs,array}
\usepackage{calc} % for calculating minipage widths
% Correct order of tables after \paragraph or \subparagraph
\usepackage{etoolbox}
\makeatletter
\patchcmd\longtable{\par}{\if@noskipsec\mbox{}\fi\par}{}{}
\makeatother
% Allow footnotes in longtable head/foot
\IfFileExists{footnotehyper.sty}{\usepackage{footnotehyper}}{\usepackage{footnote}}
\makesavenoteenv{longtable}
\usepackage{graphicx}
\makeatletter
\def\maxwidth{\ifdim\Gin@nat@width>\linewidth\linewidth\else\Gin@nat@width\fi}
\def\maxheight{\ifdim\Gin@nat@height>\textheight\textheight\else\Gin@nat@height\fi}
\makeatother
% Scale images if necessary, so that they will not overflow the page
% margins by default, and it is still possible to overwrite the defaults
% using explicit options in \includegraphics[width, height, ...]{}
\setkeys{Gin}{width=\maxwidth,height=\maxheight,keepaspectratio}
% Set default figure placement to htbp
\makeatletter
\def\fps@figure{htbp}
\makeatother
\setlength{\emergencystretch}{3em} % prevent overfull lines
\providecommand{\tightlist}{%
  \setlength{\itemsep}{0pt}\setlength{\parskip}{0pt}}
\setcounter{secnumdepth}{5}
\ifLuaTeX
  \usepackage{selnolig}  % disable illegal ligatures
\fi

\begin{document}
\maketitle

{
\setcounter{tocdepth}{2}
\tableofcontents
}
\newpage

\listoftables

Table 1: Summary of the data
used\ldots\ldots\ldots\ldots\ldots\ldots\ldots\ldots\ldots\ldots\ldots\ldots\ldots\ldots6

Table 2: P-values for Seasonal Mann-Kendall
tests\ldots\ldots\ldots\ldots\ldots\ldots\ldots\ldots\ldots11 \newpage

\listoffigures 
\newpage

\hypertarget{rationale-and-research-questions}{%
\section{Rationale and Research
Questions}\label{rationale-and-research-questions}}

The maintenance of healthy and functioning water systems is critical not
just to human life, but also to the survival of countless other species
and their interconnected ecosystems. Conductivity, or the ability of
water to pass an electrical current via dissolved salts and other
minerals, is a strong indicator of salinity, a key component of water
quality. Changes in conductivity over time suggest potential salt and
mineral pollutants entering the system, and also potential salinization
over time.

For this study, we will analyze how specific conductance varies over
time. In particular, we will address the impact of winterizing the
roadways with salt on water quality. Our main research question is:
\emph{Is salting the roads a main driver in changes to water
conductivity, or are other minerals a significant factor?} Here, we
chose to analyze water quality and flow data from the Neuse River in
Kinston, North Carolina from 1976 to 2022. We chose this river due to
familiarity with the data and its proximity to an urban center (Kinston
pop = 20,398).

We used the following research questions to guide our work:

\begin{quote}
\begin{enumerate}
\def\labelenumi{\arabic{enumi}.}
\tightlist
\item
  How does specific conductance vary over time and seasonally?
\item
  Is calcium, magnesium, or sodium the driver of specific conductance?
\end{enumerate}
\end{quote}

\newpage

\hypertarget{dataset-information}{%
\section{Dataset Information}\label{dataset-information}}

\textbf{Neuse River water quality and discharge data at Kinston, North
Carolina.} The gage information comes from the United States Geologic
Survey (USGS) National Water Information Systems (NWIS) database. USGS
gage stations typically collect discharge and a subset collect water
quality data. This water quality data may include nutrient
concentrations, concentrations of chlorophyll a, specific conductivity,
and concentrations of certain ions. Since seasonal salinity trends and
their potential sources are the focus of this study, the water quality
data being examined includes specific conductivity and concentrations of
calcium, magnesium, and sodium in the water column. Specific
conductivity will be used as a proxy for salinization, and the relative
amounts of each salt ion will be examined in the hopes of identifying a
potential source of any seasonal salinity increases, in particular the
contribution of road salts to salinization.

To obtain both sets of data, the dataRetrieval package was used to
connect directly to the NWIS database and pull water quality and
discharge data without needing to download any files. Both sets of data
are pulled starting in 1976 and end at the most recent data point in the
database in 2022.

Table 1: Summary of the data used.

\begin{longtable}[]{@{}
  >{\raggedright\arraybackslash}p{(\columnwidth - 2\tabcolsep) * \real{0.43}}
  >{\raggedright\arraybackslash}p{(\columnwidth - 2\tabcolsep) * \real{0.57}}@{}}
\toprule
\begin{minipage}[b]{\linewidth}\raggedright
Dataset
\end{minipage} & \begin{minipage}[b]{\linewidth}\raggedright
Info
\end{minipage} \\
\midrule
\endhead
NeuseWQ & Water quality data collected at USGS gage 02089500 \\
NeuseFlow & Discharge data collected at USGS gage 02089500 \\
Retrieved from & USGS NWIS database with dataRetrieval package \\
Variables & Specific conductivity, calcium, sodium, \& magnesium
concentrations, sample date, discharge \\
Date Range & 1976 through 2022, wrangled to 2013 through 2022 \\
\bottomrule
\end{longtable}

\newpage

\hypertarget{exploratory-analysis}{%
\section{Exploratory Analysis}\label{exploratory-analysis}}

The first step we took in our initial exploratory analysis was to
wrangle the water quality (WQ) dataset to include only the columns of
interest. This included the sampling dates and concentrations for
specific conductance, calcium (Ca), sodium (Na), and magnesium (Mg),
which were each given separate columns. This dataset contains monthly
observations, however, not necessarily sampled on the first of each
month. We wrangled the WQ dataset to round the dates to the first of the
month to ensure that there are evenly spaced time steps across the
years, a necessary condition for time series analyses.

We plotted the specific conductance over time to visualize any gaps in
our dataset (Figure 1). We see that the WQ dataset contains many long
periods of missing data for specific conductance. Since these missing
periods frequently span across many years, we chose to look at WQ data
from 2013 through 2021 (Figure 2). There are no missing data points from
this period of time, and will therefore not require any interpolation of
this dataset.

\begin{figure}

\includegraphics{EDA-final-project_files/figure-latex/Visualize monthly WQ data-1} \hfill{}

\caption{Specific Conductance in the Neuse River}\label{fig:Visualize monthly WQ data}
\end{figure}

\begin{figure}

\includegraphics{EDA-final-project_files/figure-latex/Visualize 2013-2021 WQ data (Conductance)-1} \hfill{}

\caption{Specific Conductance in the Neuse River from 2013 through 2021.}\label{fig:Visualize 2013-2021 WQ data (Conductance)}
\end{figure}

\newpage

We are also interested in the flow of the Neuse River because this
factor may affect salinity. For example, higher discharges may dilute
any salinity and drier periods may reflect higher salt content. We
started by wrangling the flow dataset to include the parameters of
interest, sampling date and discharge (Figure 3).

\begin{figure}

\includegraphics{EDA-final-project_files/figure-latex/Discharge over time-1} \hfill{}

\caption{Discharge in the Neuse River from 2013 through 2021.}\label{fig:Discharge over time}
\end{figure}

\newpage

\hypertarget{analysis}{%
\section{Analysis}\label{analysis}}

\hypertarget{question-1-how-does-specific-conductance-vary-over-time-and-seasonally}{%
\subsection{Question 1: How does specific conductance vary over time and
seasonally?}\label{question-1-how-does-specific-conductance-vary-over-time-and-seasonally}}

\hypertarget{is-the-specific-conductivity-of-the-neuse-river-monotonically-changing-over-time}{%
\subsubsection{Is the specific conductivity of the Neuse River
monotonically changing over
time?}\label{is-the-specific-conductivity-of-the-neuse-river-monotonically-changing-over-time}}

To determine whether specific conductance varies over time, we first
constructed and decomposed the time series data. The time series
decomposition of specific conductivity shows a seasonal component in the
Neuse River (Figure 4). To test whether the Neuse River exhibits a
monotonic trend in specific conductance, we conducted a seasonal
MannKendall test. With a p-value \textgreater{} 0.05, we accept the null
hypothesis that there the specific conductance of the Neuse River
exhibits no monotonic trend (Table 2), meaning that specific
conductivity is neither increasing or decreasing over time (Figure 5).

\begin{figure}

{\centering \includegraphics{EDA-final-project_files/figure-latex/Plot of Conductivity Time Series Decomp-1} 

}

\caption{Time Series Decomposition of Conductivity in the Neuse River.}\label{fig:Plot of Conductivity Time Series Decomp}
\end{figure}

\begin{figure}

{\centering \includegraphics{EDA-final-project_files/figure-latex/Plot of Conductivity over Time with Trend-1} 

}

\caption{Specific Conductance in the Neuse River does not exhibit a monotonic trend over time.}\label{fig:Plot of Conductivity over Time with Trend}
\end{figure}

\newpage

Unsurprisingly, the time series decomposition and seasonal MannKendall
tests of calcium (Ca) (Figures 6-7), magnesium (Mg) (Figures 8-9), and
sodium (Na) (Figures 10-11) show that the these minerals also do not
exhibit monotonic trends over time (p-values in Table 2), meaning that
mineral content is also neither increasing or decreasing over time.

\begin{figure}

{\centering \includegraphics{EDA-final-project_files/figure-latex/Plot of Calcium Time Series Decomp-1} 

}

\caption{Time Series Decomposition of Calcium in the Neuse River.}\label{fig:Plot of Calcium Time Series Decomp}
\end{figure}

\begin{figure}

{\centering \includegraphics{EDA-final-project_files/figure-latex/Plot of Calcium over Time with Trend-1} 

}

\caption{Calcium in the Neuse River does not exhibit a monotonic trend over time.}\label{fig:Plot of Calcium over Time with Trend}
\end{figure}

\begin{figure}

{\centering \includegraphics{EDA-final-project_files/figure-latex/Plot of Magnesium Time Series Decomp-1} 

}

\caption{Time Series Decomposition of Magnesium in the Neuse River.}\label{fig:Plot of Magnesium Time Series Decomp}
\end{figure}

\begin{figure}

{\centering \includegraphics{EDA-final-project_files/figure-latex/Plot of Magnesium over Time with Trend-1} 

}

\caption{Magnesium in the Neuse River does not exhibit a monotonic trend over time.}\label{fig:Plot of Magnesium over Time with Trend}
\end{figure}

\begin{figure}

{\centering \includegraphics{EDA-final-project_files/figure-latex/Plot of Sodium Time Series Decomp-1} 

}

\caption{Time Series Decomposition of Sodium in the Neuse River.}\label{fig:Plot of Sodium Time Series Decomp}
\end{figure}

\begin{figure}

{\centering \includegraphics{EDA-final-project_files/figure-latex/Plot of Sodium over Time with Trend-1} 

}

\caption{Sodium in the Neuse River does not exhibit a monotonic trend over time.}\label{fig:Plot of Sodium over Time with Trend}
\end{figure}

\newpage

We also constructed and decomposed the time series for our flow dataset
to determine whether discharge could be affecting our salinization
results. The decomposition shows that discharge in the Neuse River basin
has a seasonal component and a trend (Figure 12). The seasonal
MannKendall test shows that discharge exhibits a monotonic trend over
time, with flow significantly increasing year after year (p-value =
3.368e-07) (Figure 13, Table 2). This increasing trend in flow may be
masking increases in salt content in the river, as increased flow will
dilute an increased salt content to the same concentration, hiding an
increase in salt deposition to the river.

\begin{figure}

{\centering \includegraphics{EDA-final-project_files/figure-latex/Plot of Discharge Time Series Decomposition-1} 

}

\caption{Time Series Decomposition of Discharge in the Neuse River.}\label{fig:Plot of Discharge Time Series Decomposition}
\end{figure}

\begin{figure}

{\centering \includegraphics{EDA-final-project_files/figure-latex/Plot of Discharge over Time with Trend-1} 

}

\caption{Discharge in the Neuse River does exhibit a monotonic trend over time.}\label{fig:Plot of Discharge over Time with Trend}
\end{figure}

\newpage

\begin{center}\rule{0.5\linewidth}{0.5pt}\end{center}

Table 2: P-values for Seasonal Mann-Kendall tests.

\begin{longtable}[]{@{}ll@{}}
\toprule
Measure & P-value \\
\midrule
\endhead
Conductivity & 0.25276 \\
Calcium & 0.1038 \\
Magnesium & 0.29151 \\
Sodium & 0.091911 \\
Discharge & 3.368e-07 \\
\bottomrule
\end{longtable}

After running the the Seasonal MannKendall test, we have found no
significant change in the Neuse River over time for conductivity
(p-value = 0.25276)(Table 2). However, we were able to reject the Null
for flow data, and determine there has been a significant increase in
discharge over time (p-value = 3.368e-07) (Table 2).

\newpage

\hypertarget{is-there-seasonality-in-specific-conductivity-and-flow-of-the-neuse-river}{%
\subsubsection{Is there seasonality in specific conductivity and flow of
the Neuse
River?}\label{is-there-seasonality-in-specific-conductivity-and-flow-of-the-neuse-river}}

We plotted each mineral by specific conductance and colored by the day
of the year (DOY) to visualize any seasonal patterns within a year
(Figure 14). If the specific conductance or the different minerals
exhibited seasonality, we would expect to see clusters of the same color
within the plot. Although our time series analysis showed a seasonal
component, we see that the DOY coloring shows no patterns, and thus,
specific conductance and the minerals of Ca, Mg, and Na do not have any
seasonal trend over the year. However, we do see that Na shows the
strongest correlation with specific conductance out of all the minerals
shown (stronger positive trend).

\begin{figure}

{\centering \includegraphics{EDA-final-project_files/figure-latex/Plot by DOY-1} 

}

\caption{Salts by Specific Conductivity over the Year}\label{fig:Plot by DOY}
\end{figure}

We also plotted boxplots of specific conductance and discharge for each
month to visualize any seasonality over a year (Figure 15). We would
expect to see high values of specific conductance in winter months when
salts are applied to roads before/after snowfall, and low values
specific conductance values in the summer months when road salts are not
applied. We do not see any strong seasonal specific conductance trends
in these plots. However, viewing the boxplots in conjunction with
discharge, we see that there is generally an inverse relationship
between the two. As discharge is high in winter months, specific
conductance is low, likely because the minerals are diluted.

\begin{figure}

{\centering \includegraphics{EDA-final-project_files/figure-latex/Boxplots of Specific Conductance and Discharge-1} 

}

\caption{Comparison of specific conductance and discharge over the months.}\label{fig:Boxplots of Specific Conductance and Discharge}
\end{figure}

\newpage

\hypertarget{question-2-is-calcium-magnesium-or-sodium-the-driver-of-specific-conductance}{%
\subsection{Question 2: Is calcium, magnesium, or sodium the driver of
specific
conductance?}\label{question-2-is-calcium-magnesium-or-sodium-the-driver-of-specific-conductance}}

While there is not much seasonality in the specific conductance found at
this site in the Neuse River basin, the salt ions measured can still be
examined to see if one or another appears to be driving the
conductivity. It appears as though sodium is the ion most closely
associated with the specific conductivity found at this site, as this
ion both most closely aligns with the patterns seen in the data and also
is the most abundant ion.

\begin{figure}

\includegraphics{EDA-final-project_files/figure-latex/Visualise salts and conductivity-1} \hfill{}

\caption{Specific Conductivity and Salts in the Neuse River from 2013 through 2021}\label{fig:Visualise salts and conductivity}
\end{figure}

\begin{figure}

{\centering \includegraphics{EDA-final-project_files/figure-latex/Plot by minerals by specific conductance-1} 

}

\caption{Salts by Specific Conductivity.}\label{fig:Plot by minerals by specific conductance}
\end{figure}

We also examined the correlation between each mineral ion and specific
conductivity and plotted them in figure 17 (below). Sodium is the most
closely correlated with specific conductivity (r-squared = 0.955),
followed by magnesium (r-squared = 0.854), and calcium (r-squared =
0.831). All three minerals are pretty highly correlated with specific
conductivity, indicating they all play a pretty significant role in the
salinity of the river, but sodium has a more significant impact. These
findings corroborate what we saw above in that sodium is likely more of
a driver of salinity in the Neuse than the other two minerals.

\newpage

\hypertarget{summary-and-conclusions}{%
\section{Summary and Conclusions}\label{summary-and-conclusions}}

The Neuse River at Kinston does not show strong seasonality in salinity
and appears to show stationarity in salinity levels over the study
period from 2013-2022. Specific conductance, the proxy we used to
examine salinity, showed marginal seasonal trends when examined with a
time series decomposition(Figure 4) and no discernible seasonal pattern
when plotted by the day of the year (Figure 14). Discharge similarly
shows some marginal seasonality, and also shows a slight increasing
trend. This slight increasing trend is interesting because it could hide
a similarly-scaled increase in salinity over the same time period, as
increased discharge would dilute increased salinity effectively to show
no change. Interestingly, the change over the year in salinity was
inversely correlated with discharge (Figure 15). This potentially
indicates that any seasonal change in discharge is not due to changes in
salt influx into the stream, but rather to changes in the amount of
water available to dilute the salt content already in the stream. This
would partially explain why there is not a strong seasonal component to
the data for each mineral or specific conductivity, since the variation
is due to variability in discharge rather than salt content.

Sodium appears to have the strongest impact on specific conductance,
indicating that it perhaps is the main driver of the salinity that does
exist in the Neuse River. Sodium is the most abundant ion in the water,
magnesium and calcium had lower concentration. Sodium also has a higher
correlation with specific conductivity than the other two minerals,
indicating that sodium ions are the likely driver of salinity in the
Neuse over magnesium or calcium ions.

Overall, this is good news for the Neuse, indicating that at Kinston,
which is relatively far down the watershed, upstream uses of salts do
not cause seasonal changes in the river. The location of Kinston in the
watershed has been brought up in this analysis multiple times because it
is significant. This gage site is potentially far enough downstream from
where road salt would be used to have the salinity be insignificant by
the time the water reached Kinston. This gage was chosen in part because
it actually measures the salinity data we were interested in (and in
part because we knew it was a pretty good dataset), but a future study
would be interesting to compare the data at Kinston to similarly
collected data at a gage site further upstream, like at Clayton, which
is relatively close to the major metro area around Raleigh where road
salt is also likely to be more prevalent than in smaller population
areas.

\end{document}
