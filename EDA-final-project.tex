% Options for packages loaded elsewhere
\PassOptionsToPackage{unicode}{hyperref}
\PassOptionsToPackage{hyphens}{url}
%
\documentclass[
  12pt,
]{article}
\title{Seasonal Salt Content in the Neuse River}
\usepackage{etoolbox}
\makeatletter
\providecommand{\subtitle}[1]{% add subtitle to \maketitle
  \apptocmd{\@title}{\par {\large #1 \par}}{}{}
}
\makeatother
\subtitle{\url{https://github.com/atf35/CarpenterFischerMacDonald_ENV872_EDA_FinalProject}}
\author{Atalie Fischer, Kathlyn MacDonald, Jack Carpenter}
\date{}

\usepackage{amsmath,amssymb}
\usepackage{lmodern}
\usepackage{iftex}
\ifPDFTeX
  \usepackage[T1]{fontenc}
  \usepackage[utf8]{inputenc}
  \usepackage{textcomp} % provide euro and other symbols
\else % if luatex or xetex
  \usepackage{unicode-math}
  \defaultfontfeatures{Scale=MatchLowercase}
  \defaultfontfeatures[\rmfamily]{Ligatures=TeX,Scale=1}
  \setmainfont[]{Times New Roman}
\fi
% Use upquote if available, for straight quotes in verbatim environments
\IfFileExists{upquote.sty}{\usepackage{upquote}}{}
\IfFileExists{microtype.sty}{% use microtype if available
  \usepackage[]{microtype}
  \UseMicrotypeSet[protrusion]{basicmath} % disable protrusion for tt fonts
}{}
\makeatletter
\@ifundefined{KOMAClassName}{% if non-KOMA class
  \IfFileExists{parskip.sty}{%
    \usepackage{parskip}
  }{% else
    \setlength{\parindent}{0pt}
    \setlength{\parskip}{6pt plus 2pt minus 1pt}}
}{% if KOMA class
  \KOMAoptions{parskip=half}}
\makeatother
\usepackage{xcolor}
\IfFileExists{xurl.sty}{\usepackage{xurl}}{} % add URL line breaks if available
\IfFileExists{bookmark.sty}{\usepackage{bookmark}}{\usepackage{hyperref}}
\hypersetup{
  pdftitle={Seasonal Salt Content in the Neuse River},
  pdfauthor={Atalie Fischer, Kathlyn MacDonald, Jack Carpenter},
  hidelinks,
  pdfcreator={LaTeX via pandoc}}
\urlstyle{same} % disable monospaced font for URLs
\usepackage[margin=2.54cm]{geometry}
\usepackage{longtable,booktabs,array}
\usepackage{calc} % for calculating minipage widths
% Correct order of tables after \paragraph or \subparagraph
\usepackage{etoolbox}
\makeatletter
\patchcmd\longtable{\par}{\if@noskipsec\mbox{}\fi\par}{}{}
\makeatother
% Allow footnotes in longtable head/foot
\IfFileExists{footnotehyper.sty}{\usepackage{footnotehyper}}{\usepackage{footnote}}
\makesavenoteenv{longtable}
\usepackage{graphicx}
\makeatletter
\def\maxwidth{\ifdim\Gin@nat@width>\linewidth\linewidth\else\Gin@nat@width\fi}
\def\maxheight{\ifdim\Gin@nat@height>\textheight\textheight\else\Gin@nat@height\fi}
\makeatother
% Scale images if necessary, so that they will not overflow the page
% margins by default, and it is still possible to overwrite the defaults
% using explicit options in \includegraphics[width, height, ...]{}
\setkeys{Gin}{width=\maxwidth,height=\maxheight,keepaspectratio}
% Set default figure placement to htbp
\makeatletter
\def\fps@figure{htbp}
\makeatother
\setlength{\emergencystretch}{3em} % prevent overfull lines
\providecommand{\tightlist}{%
  \setlength{\itemsep}{0pt}\setlength{\parskip}{0pt}}
\setcounter{secnumdepth}{5}
\ifLuaTeX
  \usepackage{selnolig}  % disable illegal ligatures
\fi

\begin{document}
\maketitle

{
\setcounter{tocdepth}{2}
\tableofcontents
}
\newpage

\listoftables

Table 1: Summary of the data
used\ldots\ldots\ldots\ldots\ldots\ldots\ldots\ldots\ldots\ldots\ldots\ldots\ldots\ldots..6

Table 2: P-values for Seasonal Mann-Kendall
tests\ldots\ldots\ldots\ldots\ldots\ldots\ldots\ldots\ldots11 \newpage

\listoffigures 
\newpage

\hypertarget{rationale-and-research-questions}{%
\section{Rationale and Research
Questions}\label{rationale-and-research-questions}}

The maintenance of healthy and functioning water systems is critical not
just to human life, but also to the survival of countless other species
and their interconnected ecosystems. Conductivity, or the ability of
water to pass an electrical current via dissolved salts and other
minerals, is a strong indicator of salinity, a key component of water
quality. Changes in conductivity over time suggest potential salt and
mineral pollutants entering the system, and also potential salinization
over time.

\textless\textless\textless\textless\textless\textless\textless{} HEAD
For this study, we chose to analyze how specific conductance can vary
depending on the season. In particular, we wanted to address the impact
of winterizing the roadways with salt on water quality. Our main
research question was: is salting the roads a main driver in changes to
water conductivity, or are other minerals a significant factor? Here, we
chose to analyze the Neuse River in Kinston, North Carolina over a 46
year period (1976-2022). We chose this river due to familiarity with the
data and its proximity to an urban center (Kinston pop = 20,398).
======= For this study, we will analyze how specific conductance varies
over time. In particular, we will address the impact of winterizing the
roadways with salt on water quality. Our main research question is: Is
salting the roads a main driver in changes to water conductivity, or are
other minerals a significant factor? Here, we chose to analyze water
quality and flow data from the Neuse River in Kinston, North Carolina
from 1976 to 2022. We chose this river due to familiarity with the data
and its proximity to an urban center (Kinston pop = 20,398). We also
know from living in the headwaters of this river that salt is applied to
roadways in the upper watershed in the winter (albeit infrequently).
Since the gage at Kinston is further downstream, we will examine whether
there are significant downstream effects of road salting on water
quality in the Neuse River basin. ADD MORE?
\textgreater\textgreater\textgreater\textgreater\textgreater\textgreater\textgreater{}
a839bb1c42e04809884d612d445d82ba6e9e5349

We used the following research questions to guide our work:

\begin{quote}
\begin{enumerate}
\def\labelenumi{\arabic{enumi}.}
\tightlist
\item
  How does specific conductance vary seasonally?
\item
  Is calcium, magnesium, or sodium the driver of specific conductance?
\item
  What is the likely specific conductance in the future (forcasting
  trends)?
\end{enumerate}
\end{quote}

\newpage

\hypertarget{dataset-information}{%
\section{Dataset Information}\label{dataset-information}}

\textbf{Neuse River water quality and discharge data at Kinston, North
Carolina.} The gage information comes from the United States Geologic
Survey (USGS) National Water Information Systems (NWIS) database. USGS
gage stations typically collect discharge and a subset collect water
quality data. This water quality data may include nutrient
concentrations, concentrations of chlorophyll a, specific conductivity,
and concentrations of certain ions. Since seasonal salinity trends and
their potential sources are the focus of this study, the water quality
data being examined includes specific conductivity and concentrations of
calcium, magnesium, and sodium in the water column. Specific
conductivity will be used as a proxy for salinization, and the relative
amounts of each salt ion will be examined in the hopes of identifying a
potential source of any seasonal salinity increases, in particular the
contribution of road salts to salinization.

To obtain both sets of data, the dataRetrieval package was used to
connect directly to the NWIS database and pull water quality and
discharge data without needing to download any files. Both sets of data
are pulled starting in 1976 and end at the most recent data point in the
database in 2022.

Table 1: Summary of the data used.

\begin{longtable}[]{@{}
  >{\raggedright\arraybackslash}p{(\columnwidth - 2\tabcolsep) * \real{0.43}}
  >{\raggedright\arraybackslash}p{(\columnwidth - 2\tabcolsep) * \real{0.57}}@{}}
\toprule
\begin{minipage}[b]{\linewidth}\raggedright
Dataset
\end{minipage} & \begin{minipage}[b]{\linewidth}\raggedright
Info
\end{minipage} \\
\midrule
\endhead
NeuseWQ & Water quality data collected at USGS gage 02089500 \\
NeuseFlow & Discharge data collected at USGS gage 02089500 \\
Retrieved from & USGS NWIS database with dataRetrieval package \\
Variables & Specific conductivity, calcium, sodium, \& magnesium
concentrations, sample date, discharge \\
Date Range & 1976 through 2022, wrangled to 2013 through 2022 \\
\bottomrule
\end{longtable}

\newpage

\hypertarget{exploratory-analysis}{%
\section{Exploratory Analysis}\label{exploratory-analysis}}

The first step we took in our initial exploratory analysis was to
wrangle the water quality (WQ) dataset to include only the columns of
interest. This included the sampling dates and concentrations for
specific conductance, calcium (Ca), sodium (Na), and magnesium (Mg),
which were each given separate columns. This dataset contains monthly
observations, however, not necessarily sampled on the first of each
month. We wrangled the WQ dataset to round the dates to the first of the
month to ensure that there are evenly spaced time steps across the
years, a necessary condition for time series analyses.

We plotted the specific conductance over time to visualize any gaps in
our dataset (Figure 1). We see that the WQ dataset contains many long
periods of missing data for specific conductance. Since these missing
periods frequently span across many years, we chose to look at WQ data
from 2013 through 2021 (Figure 2). There are no missing data points from
this period of time, and will therefore not require any interpolation of
this dataset.

\begin{figure}

\includegraphics{EDA-final-project_files/figure-latex/Visualize monthly WQ data-1} \hfill{}

\caption{Specific Conductance in the Neuse River}\label{fig:Visualize monthly WQ data}
\end{figure}

\begin{figure}

\includegraphics{EDA-final-project_files/figure-latex/Visualize 2013-2021 WQ data (Conductance)-1} \hfill{}

\caption{Specific Conductance in the Neuse River from 2013 through 2021.}\label{fig:Visualize 2013-2021 WQ data (Conductance)}
\end{figure}

\newpage

We are also interested in the flow of the Neuse River because this
factor may affect salinity. For example, higher discharges may dilute
any salinity and drier periods may reflect higher salt content. We
started by wrangling the flow dataset to include the parameters of
interest, sampling date and discharge (Figure 3).

\begin{figure}

\includegraphics{EDA-final-project_files/figure-latex/Discharge over time-1} \hfill{}

\caption{Discharge in the Neuse River from 2013 through 2021.}\label{fig:Discharge over time}
\end{figure}

\newpage

\hypertarget{analysis}{%
\section{Analysis}\label{analysis}}

\hypertarget{question-1-how-does-specific-conductance-vary-seasonally}{%
\subsection{Question 1: How does specific conductance vary
seasonally?}\label{question-1-how-does-specific-conductance-vary-seasonally}}

\hypertarget{after-running-the-the-seasonal-mannkendall-test-we-have-found-no-significant-change-in-the-neuse-river-over-time-for-conductivity-p-0.25276.-however-we-were-able-to-reject-the-null-for-flow-data-and-determine-there-has-been-a-significant-increase-in-discharge-over-time-p-3.368e-07.}{%
\section{After running the the Seasonal MannKendall test, we have found
no significant change in the Neuse River over time for conductivity (p =
0.25276). However, we were able to reject the Null for flow data, and
determine there has been a significant increase in discharge over time
(p =
3.368e-07).}\label{after-running-the-the-seasonal-mannkendall-test-we-have-found-no-significant-change-in-the-neuse-river-over-time-for-conductivity-p-0.25276.-however-we-were-able-to-reject-the-null-for-flow-data-and-determine-there-has-been-a-significant-increase-in-discharge-over-time-p-3.368e-07.}}

To determine whether specific conductance varies over time, we first
constructed and decomposed the time series. The time series
decomposition of specific conductivity shows a seasonal component in the
Neuse River (Figure 4). After removing the seasonal component of the
time series, a seasonal MannKendall test shows that the trend over time
generally exhibits stationarity (p-value = 0.25276) (Table 2), meaning
that specific conductivity is neither increasing or decreasing over
time.

\begin{figure}

{\centering \includegraphics{EDA-final-project_files/figure-latex/Plot of Conductivity Time Series Decomp-1} 

}

\caption{Time Series Decomposition of Conductivity in the Neuse River.}\label{fig:Plot of Conductivity Time Series Decomp}
\end{figure}

\newpage

Unsurprisingly, the time series decomposition and seasonal MannKendall
tests of calcium (Ca), magnesium (Mg), and sodium (Na) show that the
these minerals generally exhibit stationarity over time (Figures 5-7)
(p-values in Table 2), meaning that mineral content is also neither
increasing or decreasing over time.

\begin{figure}

{\centering \includegraphics{EDA-final-project_files/figure-latex/Plot of Calcium Time Series Decomp-1} 

}

\caption{Time Series Decomposition of Calcium in the Neuse River.}\label{fig:Plot of Calcium Time Series Decomp}
\end{figure}

\begin{figure}

{\centering \includegraphics{EDA-final-project_files/figure-latex/Plot of Magnesium Time Series Decomp-1} 

}

\caption{Time Series Decomposition of Magnesium in the Neuse River.}\label{fig:Plot of Magnesium Time Series Decomp}
\end{figure}

\begin{figure}

{\centering \includegraphics{EDA-final-project_files/figure-latex/Plot of Sodium Time Series Decomp-1} 

}

\caption{Time Series Decomposition of Sodium in the Neuse River.}\label{fig:Plot of Sodium Time Series Decomp}
\end{figure}

\newpage

We also constructed and decomposed the time series for our flow dataset.
The decomposition shows that discharge in the Neuse River basin has
seasonality (Figure 8). The seasonal MannKendall test shows that
discharge exhibits a non-monotonic trend over time, with flow
significantly increasing year after year (p-value = 3.368e-07, Table 2).
This increasing trend in flow may be masking increases in salt content
in the river, as increased flow will dilute an increased salt content to
the same concentration, hiding an increase in salt deposition to the
river.

\begin{figure}

{\centering \includegraphics{EDA-final-project_files/figure-latex/Plot of Discharge Time Series Decomposition-1} 

}

\caption{Time Series Decomposition of Discharge in the Neuse River.}\label{fig:Plot of Discharge Time Series Decomposition}
\end{figure}

\begin{center}\rule{0.5\linewidth}{0.5pt}\end{center}

Table 2: P-values for Seasonal Mann-Kendall tests.

\begin{longtable}[]{@{}ll@{}}
\toprule
Measure & P-value \\
\midrule
\endhead
Conductivity & 0.25276 \\
Calcium & 0.1038 \\
Magnesium & 0.29151 \\
Sodium & 0.091911 \\
Discharge & 3.368e-07 \\
\bottomrule
\end{longtable}

\begin{center}\rule{0.5\linewidth}{0.5pt}\end{center}

\newpage

We plotted each mineral by the day of the year (DOY) to visualize any
seasonal patterns within a year. We can see that there is no correlation
between each specific conductivity or salt concentration over the year.

Higher specific conductivity is correlated with higher concentrations of
each salt, as is expected since more salt ions leads to higher salinity.
However, there is no discernible pattern related to the DOY. This
corroborates our time series analysis that there is not a significant
seasonal component the salinity of the Neuse at Kinston.

\begin{figure}

\includegraphics{EDA-final-project_files/figure-latex/Plot by DOY-1} \hfill{}

\caption{Salts by Specific Conductivity over the Year}\label{fig:Plot by DOY}
\end{figure}

\newpage

\hypertarget{question-2-is-calcium-magnesium-or-sodium-the-driver-of-specific-conductance}{%
\subsection{Question 2: Is calcium, magnesium, or sodium the driver of
specific
conductance?}\label{question-2-is-calcium-magnesium-or-sodium-the-driver-of-specific-conductance}}

While there isn't much seasonality in the specific conductance found at
this site in the Neuse, the salt ions measured there can still be
examined to see if one or another appears to be driving the
conductivity. It appears as though sodium is the ion most closely
associated with the specific conductivity found at this site, as this
ion both most closely aligns with the patterns seen in the data and also
is the most abundant ion.

\begin{figure}

\includegraphics{EDA-final-project_files/figure-latex/Visualize 2013-2021 Conductivity-1} \hfill{}

\caption{Conductivity in the Neuse River from 2013 through 2021.}\label{fig:Visualize 2013-2021 Conductivity}
\end{figure}

\begin{figure}

\includegraphics{EDA-final-project_files/figure-latex/Visualise salts and conductivity-1} \hfill{}

\caption{Specific Conductivity and Salts in the Neuse River from 2013 through 2021}\label{fig:Visualise salts and conductivity}
\end{figure}

\hypertarget{question-3-what-is-the-likely-specific-conductance-in-the-future-forcasting-trends}{%
\subsection{Question 3: What is the likely specific conductance in the
future (forcasting
trends)?}\label{question-3-what-is-the-likely-specific-conductance-in-the-future-forcasting-trends}}

Are we still evaluating this?

\newpage

\hypertarget{summary-and-conclusions}{%
\section{Summary and Conclusions}\label{summary-and-conclusions}}

The Neuse River at Kinston does not show strong seasonality in salinity
and appears to show stationarity in salinity levels over the study
period. Specific conductance, the proxy used to examine salinity, showed
marginal seasonal trends when examined with a time series decomposition
and no discernible pattern when plotted by the day of the year.
Discharge similarly shows some marginal seasonality, but also shows a
slight increasing trend. This slight increasing trend is interesting
because it could hide a similarly-scaled increase in salinity over the
same time period, as increased discharge would dilute more salinity
effectively.

Sodium appears to have the strongest impact on specific conductance,
indicating that it perhaps is the main driver of the salinity that does
exist in the Neuse River. Sodium is the most abundant ion in the water,
magnesium and calcium had a much lower concentration. Sodium also
appears to have a nearly 1:1 correlation with specific conductivity, as
seen in the DOY plot where they are plotted against each other.

Overall, this is good news for the Neuse, indicating that at Kinston,
which is relatively far down the watershed, upstream uses of salts do
not cause seasonal changes in the river. The location of Kinston in the
watershed has been brought up in this analysis multiple times because it
is significant. This gage site is potentially far enough downstream from
where road salt would be used to have the salinity be insignificant by
the time the water reached Kinston. This gage was chosen in oart because
it actually measures the salinity data we were interested in (and in
part because we knew it was a pretty good dataset), but a future study
would be interesting to compare the data at Kinston to similarly
collected data at a gage site further upstream, like at Clayton, which
is relatively close to the major metro area around Raleigh where road
salt is also likely to be more prevalent than in smaller population
areas.

\end{document}
