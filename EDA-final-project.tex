% Options for packages loaded elsewhere
\PassOptionsToPackage{unicode}{hyperref}
\PassOptionsToPackage{hyphens}{url}
%
\documentclass[
  12pt,
]{article}
\usepackage{amsmath,amssymb}
\usepackage{lmodern}
\usepackage{ifxetex,ifluatex}
\ifnum 0\ifxetex 1\fi\ifluatex 1\fi=0 % if pdftex
  \usepackage[T1]{fontenc}
  \usepackage[utf8]{inputenc}
  \usepackage{textcomp} % provide euro and other symbols
\else % if luatex or xetex
  \usepackage{unicode-math}
  \defaultfontfeatures{Scale=MatchLowercase}
  \defaultfontfeatures[\rmfamily]{Ligatures=TeX,Scale=1}
  \setmainfont[]{Times New Roman}
\fi
% Use upquote if available, for straight quotes in verbatim environments
\IfFileExists{upquote.sty}{\usepackage{upquote}}{}
\IfFileExists{microtype.sty}{% use microtype if available
  \usepackage[]{microtype}
  \UseMicrotypeSet[protrusion]{basicmath} % disable protrusion for tt fonts
}{}
\makeatletter
\@ifundefined{KOMAClassName}{% if non-KOMA class
  \IfFileExists{parskip.sty}{%
    \usepackage{parskip}
  }{% else
    \setlength{\parindent}{0pt}
    \setlength{\parskip}{6pt plus 2pt minus 1pt}}
}{% if KOMA class
  \KOMAoptions{parskip=half}}
\makeatother
\usepackage{xcolor}
\IfFileExists{xurl.sty}{\usepackage{xurl}}{} % add URL line breaks if available
\IfFileExists{bookmark.sty}{\usepackage{bookmark}}{\usepackage{hyperref}}
\hypersetup{
  pdftitle={Seasonal Salt Content, in the Neuse River},
  pdfauthor={Atalie Fischer, Kathlyn MacDonald, Jack Carpenter},
  hidelinks,
  pdfcreator={LaTeX via pandoc}}
\urlstyle{same} % disable monospaced font for URLs
\usepackage[margin=2.54cm]{geometry}
\usepackage{longtable,booktabs,array}
\usepackage{calc} % for calculating minipage widths
% Correct order of tables after \paragraph or \subparagraph
\usepackage{etoolbox}
\makeatletter
\patchcmd\longtable{\par}{\if@noskipsec\mbox{}\fi\par}{}{}
\makeatother
% Allow footnotes in longtable head/foot
\IfFileExists{footnotehyper.sty}{\usepackage{footnotehyper}}{\usepackage{footnote}}
\makesavenoteenv{longtable}
\usepackage{graphicx}
\makeatletter
\def\maxwidth{\ifdim\Gin@nat@width>\linewidth\linewidth\else\Gin@nat@width\fi}
\def\maxheight{\ifdim\Gin@nat@height>\textheight\textheight\else\Gin@nat@height\fi}
\makeatother
% Scale images if necessary, so that they will not overflow the page
% margins by default, and it is still possible to overwrite the defaults
% using explicit options in \includegraphics[width, height, ...]{}
\setkeys{Gin}{width=\maxwidth,height=\maxheight,keepaspectratio}
% Set default figure placement to htbp
\makeatletter
\def\fps@figure{htbp}
\makeatother
\setlength{\emergencystretch}{3em} % prevent overfull lines
\providecommand{\tightlist}{%
  \setlength{\itemsep}{0pt}\setlength{\parskip}{0pt}}
\setcounter{secnumdepth}{5}
\ifluatex
  \usepackage{selnolig}  % disable illegal ligatures
\fi

\title{Seasonal Salt Content, in the Neuse River}
\usepackage{etoolbox}
\makeatletter
\providecommand{\subtitle}[1]{% add subtitle to \maketitle
  \apptocmd{\@title}{\par {\large #1 \par}}{}{}
}
\makeatother
\subtitle{Web address for GitHub repository}
\author{Atalie Fischer, Kathlyn MacDonald, Jack Carpenter}
\date{}

\begin{document}
\maketitle

\newpage
\tableofcontents 
\newpage
\listoftables 
\newpage
\listoffigures 
\newpage

\hypertarget{rationale-and-research-questions-kathlyn}{%
\section{Rationale and Research Questions
(Kathlyn)}\label{rationale-and-research-questions-kathlyn}}

The maintenance of healthy and functioning water systems is critical not
just to human life, but also to the survival of countless other species
and their interconnected ecosystems. We are only beginning to understand
the impact of human action on water bodies. Conductivity, or the ability
of water to pass an electrical current via dissolved salts and other
minerals, is a strong indicator of water quality. Changes in
conductivity over time suggest a potential pollutant entering the
system.

For this study, we chose to analyze how specific conductance can vary
depending on the season. In particular, we wanted to address the impact
of winterizing the roadways with salt on water quality. Our main
research question was: is salting the roads a main driver in changes to
water conductivity, or are other minerals a significant factor? Here, we
chose to analyze the Neuse River in Kinston, North Carolina over a 46
year period (1976-2022) (???, check on this bc of the NAs). We chose
this river due to familiarity with the data and its proximity to an
urban center (Kinston pop = 20,398). ADD MORE!

We used the following research questions to guide our work:
\textgreater{} 1. How does specific conductance vary seasonally?
\textgreater{} 2. Is calcium, magnesium, or sodium the driver of
specific conductance? \textgreater{} 3. What is the likely specific
conductance in the future (forcasting trends)?

\newpage

\hypertarget{dataset-information-jack}{%
\section{Dataset Information (Jack)}\label{dataset-information-jack}}

Neuse River water quality and discharge data at Kinston, North Carolina.
The gage information comes from the United States Geologic Survey (USGS)
National Water Information Systems (NWIS) database. USGS gage stations
typically collect discharge, and a subset collect water quality data as
well. This water quality data may include nutrient concentrations,
concentrations of chlorophyll a, specific conductivity, and
concentrations of certain ions. Since seasonal salinity trends and their
potential sources are the focus of this study, the water quality data
being examined includes specific conductivity and concentrations of
calcium, magnesium, and sodium in the water column. Specific
conductivity will be used as a proxy for salinization, and the relative
amounts of each salt ion will be examined in the hopes of identifying a
potential source of any seasonal salinity increases, in particular the
contribution of road salts to salinization.

To obtain both sets of data, we used the dataRetrieval package to
connect directly to the NWIS database and pull water quality and
discharge data without needing to download it first. Both sets of data
are pulled starting in 1976, and end at the most recent data point in
the database in 2022.

\begin{longtable}[]{@{}ll@{}}
\toprule
Dataset & Info \\
\midrule
\endhead
NeuseWQ & Water quality data collected at USGS gage 02089500 \\
NeuseFlow & Discharge data collected at USGS gage 02089500 \\
\bottomrule
\end{longtable}

\newpage

\hypertarget{exploratory-analysis-atalie}{%
\section{Exploratory Analysis
(Atalie)}\label{exploratory-analysis-atalie}}

The first step we took in our initial exploratory analysis was to
wrangle the water quality (WQ) dataset to include only the columns of
interest. This included the sampling dates and concentrations for
specific conductance, calcium (Ca), sodium (Na), and magnesium (Mg),
which were each given separate columns. This dataset contains monthly
observations, however, not necessarily sampled on the first of each
month. We wrangled the WQ dataset to round the dates to the first of the
month to ensure that there are evenly spaced time steps across the
years, a necessary condition for time series analyses.

We plotted the specific conductance over time to visualize any gaps in
our dataset. We see that the WQ dataset contains many long periods of
missing data for specific conductance. Since these missing periods
frequently span across many years, we chose to look at WQ data from 2013
through 2021. There are no missing data points from this period of time,
and we will therefore not require any interpolation of this dataset.

We are also interested in the flow of the Neuse River because this
factor may affect salinity. For example, higher discharges may dilute
any salinity and drier periods may reflect higher salt content. We
started by wrangling the flow dataset to include the parameters of
interest, sampling date and discharge.

\newpage

\hypertarget{analysis}{%
\section{Analysis}\label{analysis}}

\hypertarget{question-1-how-does-specific-conductance-vary-seasonally}{%
\subsection{Question 1: How does specific conductance vary
seasonally?}\label{question-1-how-does-specific-conductance-vary-seasonally}}

After running the the MannKendall test for seasonality, we have
determined that for the Nuese river there is no significant in
seasonality for conductivity (p = 0.25276). However, we can conclude
seasonality in flow data, discharge varies significantly depending on
the season (p = 3.368e-07).

\begin{longtable}[]{@{}ll@{}}
\toprule
Measure & P-value \\
\midrule
\endhead
Conductivity & 0.25276 \\
+Calcium & 0.1038 \\
+Magnesium & 0.29151 \\
+Sodium & 0.091911 \\
Discharge & 3.368e-07 \\
\bottomrule
\end{longtable}

\hypertarget{question-2-is-calcium-magnesium-or-sodium-the-driver-of-specific-conductance}{%
\subsection{Question 2: Is calcium, magnesium, or sodium the driver of
specific
conductance?}\label{question-2-is-calcium-magnesium-or-sodium-the-driver-of-specific-conductance}}

\#\#Question 3: \textless What is the likely specific conductance in the
future (forcasting trends)?\textgreater{}

\newpage

\hypertarget{summary-and-conclusions}{%
\section{Summary and Conclusions}\label{summary-and-conclusions}}

\newpage

\hypertarget{references}{%
\section{References}\label{references}}

\textless add references here if relevant, otherwise delete this
section\textgreater{}

\end{document}
